% !TeX root = ../Document.tex
\documentclass[../Document.tex]{subfiles}

\begin{document}
\section{Uvod}

U nadolazećem tekstu imat ćete priliku da se detaljnije upoznate sa teoretskim i praktičnim procesom izrade samobalansirajućeg robota. U prvih nekolika sekcija, čitaoc će se upoznati sa tehnologijama i uređajima korištenim u svrhe izrade diplomskog rada. Također, osvrnut ćemo se na osnove elektronike koje je neizbježno razumijeti da bi se shvatio praktični dio rada. Još jedna stavka o kojoj ću pisati jesu matematičke kontrolne petlje koje su jedan od glavnih aspekata samobalansirajućeg robota.\\

\noindent Osnovni cilj ovog projekta jeste uspješna izrada robota koji se može uspravno balansirati na dva točka i kretati se pomoću komandi koje će primiti putem bluetooth veze.\\

\noindent Nakon teoretskog dijela rada detaljnije ću govoriti o samoj Android aplikaciji kao i najzanimljivijem dijelu projekta - izradi robota koji se kontroliše pomoću Arduino mikrokontrolera. Bitno je naglasiti da se radi o multidisciplinarnom radu koji integriše polja matematike, fizike, elektronike, komokunikacijskih protokola, mehanike, inžinjeringa i razvoja mobilnih aplikacija.

\end{document}